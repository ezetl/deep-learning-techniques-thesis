\documentclass{beamer}
\usepackage[utf8]{inputenc}

\usepackage{utopia} %font utopia imported

\usetheme{Boadilla}
\usecolortheme{default}

%------------------------------------------------------------
%This block of code defines the information to appear in the
%Title page
\title{Entrenamiento de modelos de aprendizaje profundo mediante autosupervisión}

\author
{Rubén Ezequiel Torti López}

\institute
{
  Facultad de Matemática, Astronomía, Física y Computación\\
  Universidad Nacional de Córdoba
}

\date
{Agosto 2017}

\logo{\includegraphics[height=1.0cm]{images/UNC-footer.jpg}}

%End of title page configuration block
%------------------------------------------------------------





%------------------------------------------------------------
%gets rid of bottom navigation bars
\setbeamertemplate{footline}[frame number]{}

%gets rid of bottom navigation symbols
\setbeamertemplate{navigation symbols}{}

%gets rid of footer
%will override 'frame number' instruction above
%comment out to revert to previous/default definitions
\setbeamertemplate{footline}{}
%------------------------------------------------------------





%------------------------------------------------------------
%The next block of commands puts the table of contents at the 
%beginning of each section and highlights the current section:
%
%\AtBeginSection[]
%{
%  \begin{frame}
%    \frametitle{Table of Contents}
%    \tableofcontents[currentsection]
%  \end{frame}
%}
%------------------------------------------------------------




\begin{document}
%The next statement creates the title page.
\frame{\titlepage}
%---------------------------------------------------------
%This block of code is for the table of contents after
%the title page
%\begin{frame}
%\frametitle{Table of Contents}
%\tableofcontents
%\end{frame}
%---------------------------------------------------------




\section{intro}
%---------------------------------------------------------
\begin{frame}
\frametitle{Aprendizaje automático}
Explora algoritmos cuyo objetivo es identificar patrones en conjuntos de datos.\pause
\vfill

Supera el enfoque clásico de instrucciones estáticas, tomando decisiones basadas en datos.\pause
\vfill

Tareas demasiado complejas para programarse directamente:
\vfill

\begin{itemize}
	\item Detección y verificación de caras
	\item Autos que se manejan solos 
        \item Reconocimiento del habla
        \item Diagnósticos médicos 
\end{itemize}
\end{frame}
%---------------------------------------------------------





%---------------------------------------------------------
\begin{frame}
\frametitle{Aprendizaje automático}
\vfill
\begin{itemize}
	\item<1-> Supervisado 
	\item<2-> No Supervisado 
        \item<3-> Por Refuerzo 
\end{itemize}
\vfill
\end{frame}

%---------------------------------------------------------





\section{marco}
%---------------------------------------------------------
\begin{frame}
\frametitle{Modelos lineales}
\begin{itemize}
    \item Clasificación
    \item Función de costo y SGD 
\end{itemize}
\end{frame}
%---------------------------------------------------------




%---------------------------------------------------------
\begin{frame}
\frametitle{Modelos no lineales}
\begin{itemize}
    \item Redes neuronales
    \item Regularizacion
    \item Backpropagation
    \item Preprocesamiento de datos
    \item Transferencia de aprendizaje
    \item Redes convolucionales
\end{itemize}
\end{frame}
%---------------------------------------------------------





%---------------------------------------------------------
\begin{frame}
\frametitle{Redes Convolucionales}
Notación:
\begin{itemize}
    \item C\(k\): capa convolucional con \(k\) filtros cuadrados
    \item F\(k\): capa completamente conectada (FC) con salida de dimensión \(k\)
    \item P: \textit{Pooling}. Usualmente \textit{MAX Pooling}
    \item D: \textit{Dropout}
    \item Op para la capa de salida
\end{itemize}
\end{frame}
%---------------------------------------------------------





%---------------------------------------------------------
\begin{frame}
\frametitle{Redes Convolucionales}
\begin{figure}
    \centering
    \includegraphics[width=\textwidth]{images/net_example.pdf}
\end{figure}
\begin{itemize}
    \item Arquitectura de red convolucional: C96-P-C256-F500-Op.
    \item Luego de cada capa convolucional se utilizan unidades ReLU y luego de cada capa completamente conectada se agrega una capa de \textit{Dropout}.
\end{itemize}
\end{frame}
%---------------------------------------------------------





\section{pretext}
%---------------------------------------------------------
\begin{frame}
\frametitle{Entrenamiento mediante tareas de pretexto}
\begin{itemize}
    \item Tareas de pretexto 
    \item Autosupervisión 
    \item Redes siamesas 
\end{itemize}
\end{frame}
%---------------------------------------------------------





%---------------------------------------------------------
\begin{frame}
\frametitle{Entrenamiento mediante tareas de pretexto}
\begin{itemize}
    \item Redes siamesas 
    \item Funciones de costo aplicadas a modelos siameses 
    \item Otro enfoque para entrenar redes siamesas 
\end{itemize}
\end{frame}
%---------------------------------------------------------






%---------------------------------------------------------
\begin{frame}
\frametitle{Información odométrica para entrenar redes siamesas}
\begin{itemize}
    \item Automovimiento 
    \item SFA 
    \item Clasificación de transformaciones en pares de imágenes 
\end{itemize}
\end{frame}
%---------------------------------------------------------





\section{Experimentos}
%---------------------------------------------------------
\begin{frame}
\frametitle{Experimentos - Prueba de concepto: MNIST}
\begin{figure}
\centering
\resizebox{.9\linewidth}{!}{
\begin{tabular}{cccccc}
\includegraphics[width = 1.5in]{./images/mnist/a1.png} &
\includegraphics[width = 1.5in]{./images/mnist/a.png} &
\includegraphics[width = 1.5in]{./images/mnist/b1.png} &
\includegraphics[width = 1.5in]{./images/mnist/b.png} &
\includegraphics[width = 1.5in]{./images/mnist/c.png} &
\includegraphics[width = 1.5in]{./images/mnist/c1.png}\\
\end{tabular}
}
\label{fig:mnist-sample}
\end{figure}

\end{frame}
%---------------------------------------------------------




%---------------------------------------------------------
\begin{frame}[plain]
\frametitle{Experimentos - Prueba de concepto: MNIST}
\begin{figure}
    \centering
    \includegraphics[width=0.9\textwidth]{images/siamese-example.pdf}
\end{figure}
\end{frame}
%---------------------------------------------------------





%---------------------------------------------------------
\begin{frame}
\frametitle{Experimentos - Prueba de concepto: MNIST}
\begin{table}
\centering
\begin{tabular}{l|rrrr}
\hline
\multicolumn{1}{r}{}
& \multicolumn{4}{c}{datos entrenamiento}
& \multicolumn{1}{l}{Método}
& \multicolumn{1}{r}{100}
& \multicolumn{1}{r}{300}
& \multicolumn{1}{r}{1000}
& \multicolumn{1}{r}{10000} \\ \cline{1-5}
\hline
Desde cero & 0.42 & 0.70 & 0.82 & 0.97\\
SFA(m=10) & 0.52 & 0.71 & 0.77 & 0.82\\
SFA(m=100) & 0.58 & 0.73 & 0.80 & 0.88\\
Automovimiento & 0.75 & 0.90 & 0.92 & 0.99\\
\hline
\end{tabular}
\end{table}
\end{frame}
%---------------------------------------------------------





%---------------------------------------------------------
\begin{frame}
\frametitle{Experimentos - KITTI}
\begin{figure}
\centering
\resizebox{.9\linewidth}{!}{
\begin{tabular}{cccccc}
\includegraphics[width = 1.5in]{./images/kitti/a.png} &
\includegraphics[width = 1.5in]{./images/kitti/a1.png} &
\includegraphics[width = 1.5in]{./images/kitti/b.png} &
\includegraphics[width = 1.5in]{./images/kitti/b1.png} &
\includegraphics[width = 1.5in]{./images/kitti/c.png} &
\includegraphics[width = 1.5in]{./images/kitti/c1.png}\\
\end{tabular}
}
\label{fig:kitti-sample}
\end{figure}
\begin{itemize}
    \item Las transformaciones en X, Y y Z se obtuvieron de las anotaciones provistas por los creadores del conjunto de datos.
    \item Para las rotaciones en el eje Y se calculó el ángulo de Euler correspondiente al cambio entre dos \textit{frames}.
\end{itemize}
\end{frame}
%---------------------------------------------------------





%---------------------------------------------------------
\begin{frame}
\frametitle{Experimentos - KITTI + SUN395}
\begin{figure}
\centering
\resizebox{.7\linewidth}{!}{
\begin{tabular}{cccc}
\includegraphics[width = 1.5in]{./images/sun397/mountain_snowy.jpg} &
\includegraphics[width = 1.5in]{./images/sun397/oilrig.jpg} &
\includegraphics[width = 1.5in]{./images/sun397/nuclear_power_plant.jpg} &
\includegraphics[width = 1.5in]{./images/sun397/rock_arch.jpg} \\
\includegraphics[width = 1.5in]{./images/sun397/subway_station.jpg} &
\includegraphics[width = 1.5in]{./images/sun397/kennel.jpg} &
\includegraphics[width = 1.5in]{./images/sun397/pilothouse.jpg} &
\includegraphics[width = 1.5in]{./images/sun397/abbey.jpg} \\
\includegraphics[width = 1.5in]{./images/sun397/vegetable_garden.jpg} &
\includegraphics[width = 1.5in]{./images/sun397/music_studio.jpg} &
\includegraphics[width = 1.5in]{./images/sun397/waterfall.jpg} &
\includegraphics[width = 1.5in]{./images/sun397/building.jpg}\\
\end{tabular}
}
\end{figure}
\end{frame}
%---------------------------------------------------------





%---------------------------------------------------------
\begin{frame}
\frametitle{Experimentos - KITTI + SUN395}
Accuracy:
\begin{table}
\centering
\resizebox{\textwidth}{!}{
\begin{tabular}{l|r|r|ccccc|r|ccccc}
\hline
\multicolumn{1}{l}{Método}
& \multicolumn{1}{r}{\#preentr.}
& \multicolumn{1}{r}{\#finet.}
& \multicolumn{1}{c}{L1}
& \multicolumn{1}{c}{L2}
& \multicolumn{1}{c}{L3}
& \multicolumn{1}{c}{L4}
& \multicolumn{1}{c}{L5}
& \multicolumn{1}{r}{\#finet.}
& \multicolumn{1}{c}{L1}
& \multicolumn{1}{c}{L2}
& \multicolumn{1}{c}{L3}
& \multicolumn{1}{c}{L4}
& \multicolumn{1}{c}{L5} \\ \cline{1-14}
\hline

ALEX-1000 & 1M & 5 & 3.73 & 5.07 & 5.07 & 8.53 & 10.40 & 20 & 9.07 & 12.53 & 16.27 & 17.60 & 10.67\\
ALEX-20 & 20K & 5 & 2.93 & 1.87 & 3.73 & 5.07 & 3.20 & 20 & 6.13 & 5.33 & 5.33 & 4.53 & 5.07\\
KITTI-SFA & 20.7K & 5 & 2.13 & 3.20 & 2.40 & 1.60 & 1.87 & 20 & 4.53 & 3.73 & 2.13 & 2.40 & 2.93\\
KITTI-EGO & 20.7K & 5 & 2.93 & 1.87 & 3.20 & 5.87 & 1.33 & 20 & 6.67 & 7.47 & 9.87 & 9.33 & 4.00\\

\hline
\end{tabular}
}
\end{table}
\end{frame}
%---------------------------------------------------------





%---------------------------------------------------------
\begin{frame}
\frametitle{Experimentos - KITTI + ImageNet}
\begin{figure}
\centering
\resizebox{.75\linewidth}{!}{
\begin{tabular}{cccc}
\includegraphics[width = 1.5in]{./images/imagenet/n01531178_108.JPEG} &
\includegraphics[width = 1.5in]{./images/imagenet/n01675722_126.JPEG} &
\includegraphics[width = 1.5in]{./images/imagenet/n01753488_177.JPEG} &
\includegraphics[width = 1.5in]{./images/imagenet/n01773797_48.JPEG} \\
\includegraphics[width = 1.5in]{./images/imagenet/n02859443_1229.JPEG} &
\includegraphics[width = 1.5in]{./images/imagenet/n03000134_789.JPEG} &
\includegraphics[width = 1.5in]{./images/imagenet/n03017168_743.JPEG} &
\includegraphics[width = 1.5in]{./images/imagenet/n03085013_2149.JPEG} \\
\includegraphics[width = 1.5in]{./images/imagenet/n03089624_632.JPEG} &
\includegraphics[width = 1.5in]{./images/imagenet/n03207941_946.JPEG} &
\includegraphics[width = 1.5in]{./images/imagenet/n03218198_298.JPEG} &
\includegraphics[width = 1.5in]{./images/imagenet/n04154565_5799.JPEG} \\
\end{tabular}
}
\end{figure}
\end{frame}
%---------------------------------------------------------





%---------------------------------------------------------
\begin{frame}
\frametitle{Experimentos - KITTI + ImageNet}
Accuracy:
\begin{table}
\centering
\begin{tabular}{l|ccccc}
\hline
\multicolumn{1}{l}{Método}
& \multicolumn{1}{c}{1}
& \multicolumn{1}{c}{5}
& \multicolumn{1}{c}{10}
& \multicolumn{1}{c}{20}
& \multicolumn{1}{c}{1000} \\ \cline{1-6}
\hline

KITTI-EGO & 0.49 & 1.27 & 2.14 & 4.13 & 20.8\\
KITTI-SFA & 0.35 & 0.75 & 1.34 & 2.64 & 11.83\\
ALEXNET & 0.45 & 0.95 & 1.91 & 3.69 & 18.35\\

\hline
\end{tabular}
\end{table}
\end{frame}
%---------------------------------------------------------





%---------------------------------------------------------
\begin{frame}
\frametitle{Conclusiones}
\end{frame}
%---------------------------------------------------------





%---------------------------------------------------------
\begin{frame}
\frametitle{Trabajo a Futuro}
\end{frame}
%---------------------------------------------------------





%---------------------------------------------------------
%Highlighting text
\begin{frame}
\frametitle{Sample frame title}

In this slide, some important text will be
\alert{highlighted} beause it's important.
Please, don't abuse it.

\begin{block}{Remark}
Sample text
\end{block}

\begin{alertblock}{Important theorem}
Sample text in red box
\end{alertblock}

\begin{examples}
Sample text in green box. "Examples" is fixed as block title.
\end{examples}
\end{frame}
%---------------------------------------------------------


%---------------------------------------------------------
%Two columns
\begin{frame}
\frametitle{Two-column slide}

\begin{columns}

\column{0.5\textwidth}
This is a text in first column.
$$E=mc^2$$
\begin{itemize}
\item First item
\item Second item
\end{itemize}

\column{0.5\textwidth}
This text will be in the second column
and on a second tought this is a nice looking
layout in some cases.
\end{columns}
\end{frame}
%---------------------------------------------------------


\end{document}
